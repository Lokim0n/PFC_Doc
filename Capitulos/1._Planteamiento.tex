\chapter{Descripción del problema} \label{Capitulo: Planteamiento}

\begin{large}
\onehalfspacing %Agregar 1.5 lineas de espaciado

En este capítulo se da a conocer la problemática existente en la Universidad Autónoma del Estado de Hidalgo, Plantel CC,\footnote{CC: Ciudad del Conocimiento.} particularmente con la Unicafe\textsuperscript{®},  así mismo se muestra la solución propuesta tomando en cuenta las condiciones y necesidades específicas de los alumnos de Ciencia computacionales.

\section{Antecedentes}

Unicafe\textsuperscript{®} surge como parte de varias concesiones por parte de Universidad Autónoma del Estado de Hidalgo para mejorar el servicio de cafetería en el año 2015, esta surge como una propuesta de negocio sustentable y conveniente para la universidad, con los objetivos de acaparar la gran demanda de alimentos por parte de la comunidad y proporcionar la posibilidad de consumir alimentos con estándares altos de calidad dentro de los mismos planteles universitarios, ofreciendo en todo momento un servicio de rápido y eficiente.

\section{Problemática}

En ultimas fechas se comenta por parte de comunidad universitaria, particularmente por estudiantes de la Licenciatura en Ciencias Computacionales, que la Unicafe\textsuperscript{®} únicamente dispone de un sistema TPV sencillo, el cual no cuenta con la posibilidad de ser escalable para brinda nuevos tipos de gestión \footnote{Es un anglicismo que describe la capacidad de un negocio o sistema de crecer en magnitud. Aunque la palabra escalabilidad no existe en el diccionario de la RAE el adjetivo más cercano ampliable es de poco uso en telecomunicaciones y en ingeniería informática.}, de momento esté solo les permite realizar operaciones de manera presencial por parte de empleados de la misma, limitando así su uso.
\\
De igual forma los mismo estudiantes de la Licenciatura en Ciencias computacionales han asistido para que se les haga de su conocimiento sí existe algún portal o plataforma electrónico al cual ellos puedan recurrir para realizar sus peticiones de manera remota pues comentan no tener la posibilidad de asistir de manera presencia a realizar dicho pedido y debido a que la Unicafe\textsuperscript{®} ha negado la existencia y la no implementación de dicha herramienta que de manera constante se les ha solicitado implementen para que tengan la oportunidad de ofrecer nuevas e innovadoras formas en que los estudiantes y comunidad universitaria pueda solicitar sus pedidos, principalmente que se refieren a la preparación de alimentos fuera de su TPV. \cite{Mun} \emph{menciona que la tecnología se ha ido integrando cada vez más, debido a que es de vital importancia poder administrar y gestionar los recursos y la optimización de procesos en un restaurante como: pedidos, ingredientes, reservas, etc.}
\\
Y sugiere de manera abrupta que toda aquella empresa que brinde un servicio y trate de considerarse de prestigio esta obligado a que someterse de una u otra forma al crecimiento tecnológico, mencionando que lo peor que pueden realizar particularmente aquellas empresas que refieren sus desarrollos a la industria de restaurantera es dejar de actualizar y de estar a la vanguardia, por el hecho de que estás tienden a ofrecer un amplio catalogo de posibilidades a sus comensales.
\\
Sumado a ese factor se comenta también que gran parte del alumnado y cuerpo académico de la licenciatura cuanta con jornada completa, lo que sugiere que cuentan con un tiempo de 10 minutos entre clase y clase, negando así la posibilidad de desplazarse a la antes mencionada Unicafe\textsuperscript{®} y de manera presencial solicitar la preparación de cualquier tipo de alimento. 
\\
Ya que estos mencionan que CUALQUIER solicitud ya sea tanto de alimentos como de bebida, tiene un lapso mínimo de preparación de 20 minutos, negando así la posibilidad de que estos consuman alimentos, obligándolos a pasar por largos periodos de tiempo sin consumo de alimento alguno.

\section{Preguntas de Investigación}
\begin{itemize}
    \item ¿Qué Problemática tiene la comunidad universitaria para consumir alimento entre una clase y otra?
    \item ¿La creación de esta herramienta para la Unicafe\textsuperscript{®} realmente beneficiara a la comunidad universitaria?
    %\item ¿Es un desarrollo web es el ideal para la brindar la mejor solución al problema?
\end{itemize}\leavevmode

\section{Hipótesis}
\begin{itemize}

    \item 
\end{itemize}\leavevmode

\section{Propuesta de solución}

Debido a la antes mencionada problemática se propone el desarrollo de una herramienta que permita satisfacer las crecientes necesidades tanto de la Unicafe\textsuperscript{®} así como de la comunidad universitaria, dotando a ambos con una herramienta capaz de gestionar sus solicitudes de manera remota, permitiendo así una agilización significativa en cuanto a tiempos se refiere.

\section{Justificación}

Ante las contantes comentarios por parte del profesorado como del alumnado, de los que se han hecho eco y que a su vez ha propiciado el estado actual en el que se encuentra la Unicafe\textsuperscript{®}, resulta de especial interés conocer a través de que herramienta podría la Unicafe\textsuperscript{®} brindar a la comunidad la solución a sus comentarios, y a partir de ahí, adoptar las medidas pertinentes para prevenir un nuevo episodio similar y en caso de darse este tener la posibilidad de remediarlo a la brevedad.
\\
Por lo cual la presente investigación surge de la necesidad de dotar a toda la comunidad universitaria con una herramienta tecnológicas capaz de gestiona y administrar solicitudes del menú de la Unicafe\textsuperscript{®}, y solucionar la problemática respecto a tener que asistir de manera presencial al TPV a solicitar la compra y posteriormente la preparación de la misma.
\\
Propiciando así no solo un ahorra significativo en cuanto a los tiempos de lo cuales actualmente la comunidad universitaria no dispone e invierte en este proceso, sino que también se busca la posterior simplificación del proceso otorgando la posibilidad de omitir algunos pasos actualmente innecesarios.

\section{Objetivo General}
\begin{itemize}
    \item El objetivo de la presente investigación es dotar a la comunidad universitaria e implementar una solución tecnológica para el Unicafe\textsuperscript{®} que consista en desarrollar un sistema de información Web con aplicativo móvil, que permitan dar soporte informático al registro y seguimiento de pedidos realizados por la comunidad universitaria. El sistema de información web permitirá realizar el registro de solicitudes que utilicen la vía móvil, además de gestionar y configurar el stock que se tendrá por día. Por otro lado, el sistema móvil realizará un registro de los cliente y de las reservas con pedidos anticipados escogidos por estos mismos, las cuales serán visualizados por el administrador del sistema para dar las indicaciones pertinentes al administrador que atenderá a dicho cliente.
\end{itemize}\leavevmode
\subsection{Objetivos Específicos}
\begin{itemize}
    \item Desarrollar un software que cumpla con los últimos y más recientes estándares de calidad presentes en el mercado actual. 
    \item Dotar a la Unicafe\textsuperscript{®} con una herramienta óptima para la administración de solicitudes remotas.
    \item Posicionar esté desarrollo cómo un referente tecnológico al mencionarse sistemas de gestión de solicitudes pertenecientes a la Universidad Autónoma del Estado de Hidalgo (UAEH).
\end{itemize}\leavevmode
\section{Alcance}
\begin{itemize}
    \item El presente proyecto culminara cuando se haya puesto en producción el sistema funcional, incluyendo también la capacitación al personal, la instalación del sistema y la entrega de manuales de usuario.
\end{itemize}\leavevmode
\section*{Herramientas Utilizadas}

\end{large}
\newpage