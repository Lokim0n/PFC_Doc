\chapter{Fundamentos teóricos} \label{Capitulo: Fundamentos}

\begin{large}
\onehalfspacing %Agregar 1.5 lineas de espaciado
El presente trabajo tiene como finalidad mostrar la implementación de un sistema vía web con aplicativo móvil de gestión de pedidos en línea para el Unicafe\textsuperscript{®}. Es importante resaltar que se realizó un análisis de los principales problemas del sector de restaurantes.

\section{Marco Teórico}

El objetivo de la presente investigación es implementar una solución tecnológica que consista en desarrollar un sistema de información Web con aplicativo móvil, que permitan dar soporte informático al registro y seguimiento de pedidos realizados por el cliente, además de almacenar información del cliente para poder fidelizar a dichos clientes por su preferencia. El sistema de información web permitirá realizar el registro de solicitudes que utilicen la vía móvil, además de gestionar y configurar el stock que se tendría por día.
\\
Por otro lado, el sistema móvil realizará un registro del cliente y de las solicitudes anticipadas que este realice, las cuales serán visualizados por el administrador del sistema para dar las indicaciones correspondientes al administrador que atenderá a dicho cliente.
\\
Por lo cual se llego a la estimación de que un sistema móvil facilitara el marketing de la Unicafe\textsuperscript{®} a la hora de ofrecer su amplio catálogo de productos.

\subsection{Plataformas o aplicaciones enfocada a la Web}

Desde la perspectiva de un usuario tradicional, puede ser un poco complicado percibir la diferencia entre "sitio, aplicación y plataforma web". Según el Diccionario Oxford en línea, nos enteramos que una aplicación es "un programa o conjunto de programas realizados para ayudar al usuario de un ordenador o dispositivo electrónico a realizar o procesar una tarea específica". 
\\
\\
Por lo tanto, se podría decir que una página Web es la unidad básica de representación dentro del World Wide Web (WWW), a la par una aplicación web es aquella página en la que el usuario interactúa en forma un poco más personalizada, para llevar a cabo una tarea específica, mientras que una plataforma web no es solo un sitio web, sino que es un conjunto de herramientas y tecnologías para construir sitios y aplicaciones web.
\\
\\
Ayudado de un webmaster \footnote{ Webmaster: Termino usado a mediados de los 80´s para describir a aquella persona encargada de administrar, armar y verificar el correcto funcionamiento de una pagina web}, una plataforma web podía incluir algunos elementos adicionales las cuales permitian crear sitios web diseñados a medida y necesidades, tales como carros de compras, convertidores, enlazadores de redes sociales, formularios, portafolios de proyectos, verificación de datos, entre muchos otros, incluso proporcionar una solución específica para la necesidad del cliente y/o el mercado.

\subsection{Antecedentes y Evolución de las Plataformas web}

Las Plataformas Web poco a poco han revolucionado la forma de utilizar internet, aumentando el contenido de las páginas con texto estático (texto que no evoluciona, sino que permanecen como es) a un contenido rico e interactivo, por lo tanto escalable. Hace algunos años, los sitios web no eran mucho más que folletos digitales. Actualmente los sitios son más grandes y complejos.
\\
\\
El concepto de la aplicación web no es nuevo. De hecho, uno de los primeros lenguaje de programación para el desarrollo de aplicaciones web fue "Perl".  Este fue inventado por Larry Wall en 1987 antes de que internet se convirtiera en accesible para el público en general. Pero fue en 1995 cuando el programador Rasmus Lerdorf puso a disposición el lenguaje PHP (Hypertext Preprocessor) que es un lenguaje de código abierto para el desarrollo web y que puede ser incrustado en documentos HTML, con lo que todo el desarrollo de aplicaciones web realmente despegó.
\\
\\
Meses más tarde de la llegada de PHP, Netscape, el navegador web más antiguo y popular, anunció una nueva tecnología, JavaScript, lo que permitió a los programadores de aquella epoca cambiar a una forma un poco más dinámica el contenido de sus páginas, que había sido hasta el momento texto estático. 
Por lo cual esta nueva tecnología permitió dar un nuevo enfoque al desarrollo de aplicaciones Web. Por ejemplo, uno de los mayores exponentes en hacer uso de este lenguaje en la actualidad es Google, al mostrar los resultados de nuestras búsquedas dentro del navegador este hace un uso intensivo de JavaScript. 
\\
\\
Un año después, en 1996, dos desarrolladores, Sabeer Bhatia y Jack Smith lanzaron Hotmail, un servicio de correo en línea, innovador en cuanto a su entorno gráfico, el cual permita que permitía (por primera vez para el público en general), acceder y consultar correos electrónico mediante accesos remotos.
\\
Posteriormente llego la conocida plataforma Flash,  la cual permitía  añadir contenido interactivo dentro de los sitios web de la época. Flash o más tarde conocido como Shockwave Flash hizo su aparición en 1997. Más tarde, esta se convertiría en una plataforma para el desarrollo de aplicaciones web interactivas.
\\
\\
En 1998 la compañía Google desarrolló su primer motor de búsqueda en línea que, por su nueva forma de indexar páginas web, facilitaba enormemente la búsqueda de información en internet. 
\\
\\
A principios de 2001, poco después de la explosión de la burbuja de internet, Wikipedia se lanzó como un sub-proyecto de Nupedia, una enciclopedia en línea tradicional. Para desarrollar su plataforma, se utiliza un tipo de Wikipedia de la aplicación web denominada "wiki", que permite a cualquier usuario agregar contenido. Las contribuciones no se hicieron esperar, y al final del primer año de funcionamiento, Wikipedia ya contaba con 20000 páginas en 18 idiomas. 
\\
\\
Posteriormente en el año de 2004. En una conferencia  sobre la nueva Web 2.0 a cargo de John Battelle y Tim O'Reilly, el concepto de "web como plataforma" fue mencionado por primera vez. Esta innovación allanó el camino para futuras aplicaciones web, es decir, un software que aprovecha las ventajas de la conexión a internet y que se desvían del uso tradicional del escritorio. 
Posteriormente en eses mismo año fue el lanzamiento de Facebook, abierto sólo a los estudiantes. Con un millón de suscriptores a finales de 2004, Facebook se ha convertido en el medio de comunicación más utilizado, con más 2,000 millones de usuarios mensuales activos. Este es el segundo sitio más visitado en el planeta y tiene la mayor cantidad de fotos compartidas por los usuarios dentro de una plataforma. 

\subsection{Tipos de Plataformas}

Por todo lo antes mencionado y con la gran flexibilidad que tienen los entornos web para el desarrollo de múltiples actividades podemos definir que en la actualidad existe una multitud de plataformas o conjunto de subsistemas para crear y poner nuevos funcionamientos dentro de nuestros sitios web. 


\subsection{Lenguajes de programación WEB}

Los lenguajes de programación web, ha ido evolucionando progresivamente a medida que se incrementaban las funcionalidades cada vez más complejas que el mercado demandaba a los sitios webs. Las páginas webs estáticas, en las que sólo se requería de código HTML, evolucionaron hacia las páginas dinámicas, en las que ya se necesitaba una aplicación en el servidor, que interactuara con las bases de datos y construyera la página web según las peticiones del internauta.
\\
\\
\begin{itemize}
\item Lenguajes del lado cliente (Front-end)
    Son aquellos lenguajes que son asimilados directamente por el navegador y no necesitan pre tratamiento.
\item Lenguajes del lado servidor (Back-end)
    Son aquellos lenguajes que se ejecutan por el propio servidor y son enviados al cliente en un formato claro para él.
\end{itemize}\leavevmode

\subsection{Lenguajes Front-end}

\subsubsection{Lenguaje HTML} 
(HyperText Markup Language): apareció por primera vez en 1991 en el lanzamiento de la Web. Su función es la gestión y organización del contenido. Así que en HTML puedes escribir lo que deseas mostrar en la página: texto, enlaces, imágenes.
\\
\\
Se podría decir: Este es mi título, este es mi menú, aquí está el texto principal de la página, aquí hay una visualización de la imagen, etc.
\\
\subsubsection{Lenjuaje CSS}

(Cascading Style Sheets) significan «Hojas de estilo en cascada» y parten de un concepto simple pero muy potente: Aplicar estilos (colores, formas, márgenes, etc...) a uno o varios documentos (generalmente documentos HTML, páginas webs) de forma masiva.
\\
\\
Se le denomina estilos en cascada porque se aplican de arriba a abajo (siguiendo un patrón denominado herencia que trataremos más adelante) y en el caso de existir ambiguedad, se siguen una serie de normas para resolverla.
\\
\\
La idea de CSS es la de utilizar el concepto de separación de presentación y contenido, intentando que los documentos HTML incluyan sólo información y datos, relativos al significado de la información a transmitir (el contenido), y todos los aspectos relacionados con el estilo (diseño, colores, formas, etc...) se encuentren en un documento CSS independiente.
\\
\subsubsection{Lenguaje JavaScript}

Se utiliza principalmente del lado del cliente aunque se puede utilizar del lado del servidor. Actualmente y gracias a tecnologías como AJAX es utilizado para enviar y recibir información del servidor.

\begin{itemize}
    \item Como principales ventajas, tenemos que destacar que es un lenguaje de scripting seguro y fiable, cuyos scripts tiene capacidades limitadas, debido a la seguridad.
    \item Como desventajas, podríamos mencionar que el código debe descargarse por completo y es visible por cualquier usuario.
\end{itemize} \leavevmode

\subsection{Lenguajes Back-end}

\subsubsection{Lenguaje PHP}

Es un lenguaje enfocado en la creación de webs dinámicas. Sus scripts son interpretados por el servidor y genera código HTML. Requiere Apache o IIS con librerías de PHP. Hereda su sintaxis de C, Java y Perl.
\\
\\
Como principales ventajas, hemos de decir que es un lenguaje fácil de aprender y muy rápido. Soporta la orientación a objetos  y utiliza un lenguaje multiplataforma. Además, puede conectarse con una gran cantidad de base de datos: MySQL, Oracle.
\\
\subsubsection{Lenguaje Java}

Fue diseñado específicamente para tener tan pocas dependencias de implementación como fuera posible. 
\\
\\
Su intención es permitir que los desarrolladores de aplicaciones escriban el programa una vez y lo ejecuten en cualquier dispositivo (conocido en inglés como WORA, o "write once, run anywhere"), lo que quiere decir que el código que es ejecutado en una plataforma no tiene que ser recompilado para correr en otra.
\\

\section{Estado del Arte}

 A continuación, se muestran algunos trabajos de investigación relacionados con la implementación de tanto de aplicaciones móviles como de algunos sitios web que tienen relación a la propuesta de este trabajo.


González, María & Saraza, Joel (2014) con la tesis: Implementación de un Sistema vía web con aplicación móvil para la reserva y pedidos en línea de restaurantes. Tesis para optar el título de Ingeniero de Sistemas. Universidad de San Martín de Porres, Lima - Perú. 
\\
\\
El presente trabajo tiene como finalidad mostrar la implementación de un sistema vía web con aplicativo móvil de reservas y pedidos en línea de restaurantes. Es importante resaltar que se realizó un análisis de los principales problemas de una empresa mediana del sector de restaurantes.
\\
El objetivo de la presente investigación es implementar una solución tecnológica que consista en desarrollar un sistema de información Web y con aplicativo móvil, que permitan dar soporte informático al registro y seguimiento de las reservas con pedidos realizadas por el cliente, además de almacenar información del cliente para poder fidelizar a dichos clientes por su preferencia. El sistema de información web permitirá realizar el registro de los clientes y reservas que utilicen la vía telefónica o el correo electrónico, además de configurar el stock de cantidad de reservas que se tendría por día. Por otro lado, el sistema móvil realizará un registro del cliente y de las reservas con pedido anticipado escogido por el cliente, las cuales serán visualizados por el encargado de reservas para dar las indicaciones correspondientes al mozo que atenderá a dicho cliente. 
\\
Se llegó a la conclusión de que el sistema móvil facilita el marketing del restaurante a la hora de ofrecer todos sus platillos el cual permite ganarse con el tiempo la fidelización de los clientes. 
\\
\\
Pérez, Christian 2015 con la tesis Análisis, diseño e implementación de una guía gastronómica para la administración y ubicación de restaurantes en entorno web. Tesis para optar el título de Ingeniero de Sistemas. Pontificia Universidad Católica del Perú, Lima - Perú. 
\\
\\
En el actual contexto peruano, la gastronomía peruana es un mercado en constante movimiento tanto de alzas como de bajas. Esto quiere decir, que, así como se van formando nuevos negocios de restaurantes, también van desapareciendo otros. Asimismo, algunos de estos permiten agilizar su ubicación a través de portales web, de tal forma que facilita esta actividad para los consumidores. 22 Frente a este contexto, aparece el problema de la dificultad de la ubicación de restaurantes. Por un lado, la distribución de portales web a lo largo de Internet, desemboca en una falta de centralización, lo cual no permite a los consumidores ubicar fácilmente un restaurante que se acomode a sus necesidades. Por otro lado, no todos los restaurantes cuentan con un portal web propio, lo cual genera una falta de medios de comunicación entre los restaurantes y clientes. En este proyecto de fin de carrera, se brindará una propuesta de solución tanto al problema de descentralización de los restaurantes en Internet como a la falta de medios de comunicación entre restaurantes y consumidores. Para desarrollar esta solución se abarcará el análisis, diseño e implementación de un sistema de información en base a metodologías y procedimientos de ingeniería de software. Asimismo, se realizará el desarrollo de algoritmos para brindar soporte a algunas funcionalidades del sistema. 
\\
Se llegó a la conclusión al conseguir implementar un prototipo funcional del sistema de información que el principal aporte de este ha sido poder demostrar ser una alternativa de solución a la problemática planteada, la cual se centra en la falta de mecanismos de ubicación de restaurantes y en la falta de un artefacto que apoye en el análisis de los comentarios. Así, se logró crear un espacio en el que se mantenía un registro de restaurantes y comensales, en el cual estos últimos son capaces de encontrar los locales que se adecúen mejor a sus necesidades y criticarlos para generar información neutral con respecto a los servicios que bridan. Asimismo, este prototipo permitió integrar los mecanismos de ordenamiento y de análisis de texto que se mencionaron previamente. 
\\
\\
Pajuelo Anibal, Maco Jose, Chavez Jorge & Leandro Marlon 2015 con la tesis: Sistema para reservas online en restaurantes. Tesis para optar el título de Ingeniero de Sistemas. Universidad Peruana de Ciencias Aplicadas, Lima – Perú. 
\\
\\
Perú vive un entorno favorable de crecimiento en su economía desde los últimos 20 años a pesar de su desaceleración en los 03 últimos años, estimándose que el crecimiento continúe en los próximos años. El sector servicios, donde se ubica el rubro de restaurantes, tiene un gran aporte en el crecimiento económico del PBI y mantiene incrementos anuales mayores al 6\%. Todo ello favorecido por el importante desarrollo de nuestra gastronomía, que en los últimos años ha logrado un amplio reconocimiento en el mundo y que incentiva al consumo interno de la población, así como la oferta en la apertura de restaurantes. Por otro lado, el crecimiento de la infraestructura y la tecnología de información vienen posibilitando el desarrollo de nuevos tipos de negocio y marketing, existiendo aplicaciones para teléfonos móviles que permite a los usuarios llamar a taxis, descargar películas, comercializar bienes y otros. Esta corriente tecnológica es una buena oportunidad para el desarrollo de nuestra propuesta de negocio, que propone ser una alternativa para el público usuario que utiliza aplicaciones móviles para permitirles realizar reservas de mesas en los restaurantes de manera fácil, gratis, segura, y que adicionalmente les genera beneficios económicos. 
\\
En ese sentido, nuestra propuesta modifica la manera tradicional de asistir a restaurantes sin realizar reserva o tomando mucho tiempo en realizarla, así como en otros casos minimiza los tiempos de espera en el acceso a las mesas. Asimismo, nuestra visión se enfoca en ser reconocida en el servicio de reservas en restaurantes, a través de la innovación tecnológica. Para lograr dicho propósito nos enfocamos en un segmento de la comunidad universitaria que tiene preferencia por salir a comer y que usan teléfonos inteligentes (smartphones).
Esto nos permitirá diferenciarnos de la competencia existente que no han logrado posicionarse en el mercado local dado que el usuario final desconoce su existencia y oferta. Esto ha sido corroborado con el estudio de mercado desarrollado a nivel cualitativo (focus group, entrevista a expertos) y a nivel cuantitativo (encuestas personales), que nos ha permitido conocer lo que esperan nuestros clientes (restaurantes) así como los usuarios finales.
\\
Se llegó a la conclusión de que el sostenido crecimiento económico del Perú, el boom de nuestra gastronomía, las mejoras de la infraestructura de las TICs y la alta penetración de los teléfonos inteligentes en la población de las principales ciudades del Perú son factores favorables que incentivan al desarrollo de nuestra propuesta de negocio servicio en línea para la reserva de mesas en restaurantes. 
\\
\\
Burgos, Carlos (2015) con la tesis: Desarrollo de un sistema web para la gestión de pedidos en un restaurante aplicación a un caso de estudio. Tesis para optar el título de Ingeniero de Sistemas. Escuela Politécnica Nacional, Ecuador – Quito. 
\\
\\
En Quito existen una gran cantidad de restaurantes entre los cuales están los restaurantes gourmet, de comida rápida y especializada. 24 Los restaurantes en los cuales se enfocará este proyecto son los gourmets, debido a que son los más aptos para instalar sistemas que automaticen sus procesos, ya que cuenta con la infraestructura adecuada para la instalación de equipos computacionales. En los restaurantes gourmet el costo a de acuerdo al servicio y la calidad de los platos que se consumen. El servicio, la decoración, la ambientación, comida y bebidas son cuidadosamente escogidos. Así mismo, para mejorar la atención a sus clientes se requiere que: 
\begin{itemize}
    \item Los alimentos sean cocinados al momento, por lo que es necesario tener la información de los pedidos lo más pronto posible. 
    \item El cliente observe el listado de pedidos que ha realizado y el costo total de los mismos. 
    \item Sé más ágil el proceso de pagos, por lo que se requiere que pueda conocer el valor de su consumo rápidamente. 
\end{itemize}\leavevmode
Actualmente los restaurantes gourmet de Quito tiene muchas exigencias en cuanto a dar un buen servicio, como por ejemplo que el cliente se sienta cómodo al realizar un pedido, esto muchas veces no se da debido a que los meseros no se abastecen en atender rápidamente a las mesas, además de que se toman las órdenes manualmente para después ir a la cocina y dar a conocer el pedido realizado por el cliente. De esta manera, el proceso lleva mucho tiempo y más cuando está el restaurante lleno. 
\\
Para resolver la problemática presentada, se propone el desarrollo de un sistema web para la gestión de pedidos en un restaurante tipo gourmet, al cual se lo ha denominado SYPER (Sistema de Pedidos para Restaurantes), mismo que permitirá gestionar los pedidos de una manera rápida, segura y amigable con el cliente. \\
Se llegó a la conclusión de que la metodología XP fue un pilar muy importante a lo largo de todo el proyecto ya que al enfocarse en la funcionalidad del sistema se logró reducir errores y mejorar la calidad del mismo y utilizar MVC \footnote{MVC: Modelo-Vista-Controlador} para la estructuración del código fuente, ayudó a tener un código ordenado y de fácil entendimiento al momento de realizar un cambio o corregir un error.
\\
\\
Conde, Daniela (2012) con la tesis: Desarrollo de un sistema web para el manejo de reservaciones a través del portal web de tudescuenton.com. Tesis para optar el título de ingeniero en computación. Universidad Simón Bolívar, Venezuela – Sartenejas. 
\\
\\
TuDescuentón.com es un portal web que se encarga de vender cupones de hasta un 90\% de descuento a los usuarios registrados en su página. En años pasados, el portal decidía ampliar su modelo de negocio incursionando en los sistemas de reservas para restaurantes vía web. De ahí nace la idea de TuDescuentón Reservas, un sistema que le permitirá a los usuarios de TuDescuentón.com realizar reservaciones en los restaurantes afiliados por un monto fijo, logrando así una mesa segura a la hora deseada en el restaurante de su preferencia y con porcentaje de descuento de un 25\% sobre el consumo total. Como resultado de este proyecto se elaboró un sistema que permite a los usuarios registrados en el portal web de TuDescuentón.com, buscar restaurantes por nombre, ubicación, especialidad culinaria, más reservados, según la fecha de la reservación, disponibles por día, entre otros, y realizar reservaciones en los mismos. 25 El uso de un CMS como Drupal, si bien ayuda en algunos aspectos (manejo de usuarios, modularidad del código, manejo de la base de datos, etc.), también frena la velocidad del desarrollo en otros. Disminuye la velocidad de desarrollo ya que su curva de aprendizaje es muy inclinada y amplia gama de posibilidades lo hace complejo.
\\
Por tanto, un punto de importante que se debe tocar, es que antes de desarrollar cualquier proyecto hay que evaluar que tan beneficioso podría ser el uso de un framework o un CMS para su desarrollo. Puesto que no siempre el uso de estos genera facilidad ni agiliza el proceso. Uno de los puntos más difíciles, fue el desarrollo del tema del sistema, es decir, implantar el diseño realizado por la diseñadora en la página web. Sin embargo, cabe destacar que Drupal, es una herramienta muy poderosa, por lo que se considera es importante conocerla y manejarla, cosa que no se hace, a profundidad, en 5 meses.

\section{Uso y Justificación de la Metodología}

\end{large}
\newpage
