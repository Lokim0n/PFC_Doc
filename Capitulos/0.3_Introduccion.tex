\chapter*{Introducción}
\addcontentsline{toc}{chapter}{Introducción}

\begin{large}
\onehalfspacing %Agregar 1.5 lineas de espaciado


\section*{Estructura del documento}

\begin{itemize}
    \item El capítulo \ref{Capitulo: Planteamiento}, denominado Marco Teórico, aborda conceptos y descripciones de los elementos requeridos para el desarrollo del esqueleto de la plataforma.
    \item El capítulo \ref{Capitulo: Fundamentos} presenta el Estado del Arte, el cual cita una variedad de trabajos relacionados al que se plantea, los fundamentos teóricos necesarios para comprender los aspectos más relevantes de esta investigación.
    \item En el capítulo \ref{Capitulo: Desarrollo} describe de manera extensa todos los aspectos necesarios en el uso y desarrollo de la metodología, así mismo ejemplifica minuciosamente el desarrollo e implementación de la herramienta.
    \item El capítulo \ref{Capitulo: Implementación} muestra la validación e implementación en el campo de aplicación, de algunas herramienta, se incluyendo también el manual de usuario.
    \item El capítulo \ref{Capitulo: Resultados} muestra los resultados obtenidos mediante las encuestas realizadas.
    \item Finalmente el capítulo \ref{Capitulo: Conclusiones}, muestra el impacto obtenido al haber desarrollado este trabajo, así como el seguimiento a futuro del mismo.
\end{itemize} \leavevmode

\end{large}
\newpage
